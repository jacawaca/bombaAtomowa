\noindent
\begin{algorithm}
\caption{Algorytm na ruch atomów "w symulacji"}
\begin{algorithmic}[1]
\State $dt \gets 0.1, t \gets 0$
\State $\text{allAtom} \gets \text{UserInput}$
\State $\text{Atomy[allAtom]}\gets {0,0,0} $
\State $\text{isWorking}\gets \text{UserInput}$
\State $\text{Neutrony*}\gets 0$
\State Określenie, które punkty siatki będą zajęte przez aktywne jądra.
\State zapełnij(Atomy[])

\While{$\text{isWorking}=\boldsymbol{true}$}
    \State Pętla po wszystkich atomach, żeby określić, które ulegą samoistnemu rozpadowi oraz generacja neutronów.
    \For{i: Atomy}{$i++$}
        \If{$\text{Atomy[i].czyRozpadl}=\text{false}$ oraz czyRozpadnie()=true}{ $\text{Atomy[i].czyRozpadl}=\text{true}$}
            \For{i: 1-N}
            \State new neutron=generuj(Atom[i])
            \State Neutrony.add(neutron)
            \EndFor
            \State Energia += energiaRozpadu()
            \State $\text{newNeutron} += N$
        \EndIf
    
    \EndFor
    \State liczMoc(newNeutron)
    \State Czyścimy: $\text{newNeutron} =0$
    \State ruszajAndWybuchaj(neutrony,dt) 
    \State $t += dt$
    
\EndWhile

\end{algorithmic}
\end{algorithm}